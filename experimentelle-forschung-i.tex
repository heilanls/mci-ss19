\section{Experimentelle Forschung I -- Wie führe ich eine Studie durch?}
		\subsection{Forschungsfrage finden}
			Was man beachten sollte:
			\begin{itemize}
				\item Neues System soll entwickelt werden $ \longrightarrow $ welche menschlichen Fähigkeiten und Einschränkungen muss man beachten?
				\item Neues System wurde entwickelt $ \longrightarrow $ Wurden die Designziele erreicht? (usability testing)
			\end{itemize}
		
		\subsection{Hypothesen aufstellen}
			Hypothese = vorläufige Antwort, die mit der Wirklichkeit abgeglichen wird
			
			\paragraph{Entwicklung von Hypthesen}Man kommt auf Hypothesen durch
				\begin{itemize}
					\item Ableiten aus Forschungsliteratur
					\item Alltagserfahrungen
					\item Explorative Studien
					\item (Wunschdenken, z.B.: 'Ein neues System soll besser sein als XYZ')
				\end{itemize}
			
			\paragraph{Eigenschaften einer guten Hypothese:}
				\begin{itemize}
					\item Ist empirisch überprüfbar
					\item Generalisierbarkeit, d.h. Relevanz über den Versuchskontext hinaus
					\item Wurde vor der Überprüfung aufgestellt (Pfeil-Beispiel aus der VL)
					\item (grundsätzliche) Falsifizierbarkeit\Biglb
				\end{itemize}
			
		\subsection{Variablen}
			Eine Hypothese postuliert eine Beziehung zwischen mindestens zwei Variablen ('Variable A sorgt dafür, dass mit Variable B was passiert.'). Es gibt zwei Typen von Variablen:
			\begin{itemize}
				\item Beobachtungsnah/konkret/manifest: mit denen kann man direkt arbeiten
				\item Beobachtungsfern/latent/abstrakt: muss man erst operationalisieren, also so umformulieren, dass sie messbar sind
			\end{itemize}
			
			\paragraph{Operationalisierung:} Macht schwammige Aussagen/abstraktes theoretisches Konstrukt messbar. Beispiele:
				\begin{itemize}
					\item 'guter Fahrer' = Anzahl unfallfreier km
					\item 'xy verbessert Nutzerfreundlichkeit der Schnittstelle' = Zufriedenheitsrate in einem Fragebogen
					
					\item 'xy erhöht das Bewusstsein des Fahrers für die aktuelle Fahrsituation' = Reaktionszeit auf Ereignisse in z.B. einem Fahrsimulator \\
				\end{itemize}
			
			
		\subsection{Versuchsplan erstellen}
			Elemente eines experimentellen Designs:
			\begin{itemize}
				\item Typen von Variablen im Experiment ((un-)abhängig, Störvariable)
				\item Faktor und Level
				\item Bedingung/Kondition
				\item Durchgang/Trial
			\end{itemize}
		
			\subsubsection{Typen von Variablen}
				\paragraph{(Un-)Abhängige Variablen} Im Falle experimenteller Forschung postuliert die Hypothese eine Beziehung zwischen (einer) unabhängigen und (einer) abhängigen Variablen:
				\fig{0.5}{res/typen-von-variablen.jpg}
				
				Die unabhängige Variable (UV) kann Experimentator aktiv verändert werden bzw. wird kontrolliert. Beispiel: 'Die \textbf{Art der Nutzerschnittstelle} im Auto führt zu aggressivem Fahrverhalten.'. Typische unabh. Variablen im MCI-Bereich:
				\begin{itemize}
					\item Technologie: unterschiedliche Arten v. Geräten, versch. GUIs
					\item Nutzereigenschaften: Alter, Computererfahrung, \ldots
					\item Kontext d. Interaktion: physikalische (Beleuchtung, Lärm, Bewegung, \ldots) und soziale Faktoren (Anzahl der Menschen i.d. Umgebung,\ldots)
				\end{itemize} 
				Die abhängige Variable ist das, was gemessen wird. Beispiel: 'Die Art der Nutzerschnittstelle im Auto führt zu \textbf{aggressivem Fahrverhalten}'. Typische abh- Variablen im MCI-Bereich:
				\begin{itemize}
					\item Effizienz, z.B. task completion time
					\item Genauigkeit, z.B. Fehlerrate
					\item Subjektive Zufriedenheit
					\item Erlernbarkeit
					\item Physische und kognitive Belastung
				\end{itemize}
			\pagebreak
				\paragraph{Störvariablen (confound variables)} sind nicht Teil der Hypothese, beeinflussen aber die abhängige Variable. Beispiel: 75\% der Probanden bevorzugen die Maus gegenüber dem Trackpad, aber alle Linkshänder haben beides gut bewertet, alle Rechtshänder haben beides als schlecht bewertet. Dann ist die Verteilung der Linkshänder unter den Probanden entscheidend für die durchschnittl. Bewertung des Eingabegeräts:
				\figwithcap{0.5}{res/stoervariablen.jpg}{\textit{Dadurch, dass mehr Linkshänder die Maus bewertet haben, wird die durchschnittl. Bewertung der Maus besser}}
			
			
			\subsubsection{Faktor und Level}
				\paragraph{Faktor}Der Faktor ist ein Indikator für die Anzahl der unabhängigen Variablen:\\
				eine UV = unifaktorieller Versuchsplan; \\ zwei/drei/viele UVs = zwei-/drei-/multifaktorieller Versuchsplan
				
				\paragraph{Level} Jeder Faktor (also jede UV) besteht aus mehreren Levels. 'Level' ist hier zu verstehen als Variation einer UV:
					\begin{itemize}
						\item Eingabetechnik: Maus, Trackpad, Trackpoint
						\item Schwierigkeit: einfach, mittel, schwer
						\item Zielgröße: 1/2/3cm
					\end{itemize}
				
			\subsubsection{Kondition}
				Die Kondition = Die Levels der UV oder bei multifaktoriellen Designs alle Kombinationen der
				Levels aller UVs:
				\fig{0.5}{res/kondition.jpg}
				
			\subsubsection{Durchgang/Trial}
				Bei manchen Experimenten wird eine Kondition mehrfach pro Proband durchgeführt.
			
			
			
			
			
			
			
			
			
			
			
			
			
			
			
			
			