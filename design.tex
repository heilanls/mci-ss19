\section{Design}
	\subsection{Gestalttheorie}
		\subsubsection{Die 4 Gestaltprinzipien}
			\begin{enumerate}
				\item Emergenz: Die Herausbildung von neuen Eigenschaften oder Strukturen eines Systems
				infolge des Zusammenspiels seiner Elemente

				
				\item Reifikation: Mentaler Prozess der Vergegenständlichung -- das Gehirn füllt Informationslücken auf, da Informationen in unserer Umwelt oft unvollständig sind.
				
				\item Invarianz: Wir erkennen ein Objekt, auch wenn es sich visuell anders darstellt; unabhängig von Rotation, Transformation, Skalierung, \ldots
				
				\item Multistabilität: Mehrere Interpretationen der Umwelt sind möglich; aber zu einem Zeitpunkt ist immer nur eine Interpretation wahrnehmbar \biglb
			\end{enumerate}
		
		\subsubsection{Gruppierungsgesetze -- 'Gruppierung durch \ldots'}
			\paragraph{1.Nähe:} Objekte, die näher beieinander liegen, werden als Gruppe wahrgenommen
				\fig{0.5}{res/gesetz-der-naehe.jpg}
				
			\paragraph{2. Ähnlichkeit:} Objekte, die einander ähnlich sind, werden perzeptuell gruppiert
				\fig{0.5}{res/gesetz-der-aehnlichkeit.jpg}
		
		\pagebreak
				
			\paragraph{3. Geschlossenheit:} Elemente werden so gruppiert, das geschlossene Objekte wahrgenommen
			werden
				\fig{0.5}{res/gesetz-der-geschlossenheit.jpg}
			
			\paragraph{4. Symmetrie:} Symmetrische Interpretationen werden vorgezogen und besser wahrgenommen
				\fig{0.5}{res/gesetz-der-symmetrie.jpg}
				
			\paragraph{5. Gemeinsames Schicksal:} Elemente, die sich in dieselbe Richtung bewegen, werden als Gruppe wahrgenommen
				\figwithcap{0.5}{res/gesetz-des-gemeinsamen-schicksals.jpg}{\textit{Alle Submenu-Elemente öffnen sich nach rechts, deswegen werden Sie als Gruppe wahrgenommen}}
				
		\pagebreak
		
			\paragraph{6. Kontinuität (Gute Fortsetzung):} Linien werden so gesehen, als folgten sie einfachen Pfaden mit wenig Richtungsänderung
				\fig{0.4}{res/gesetz-der-guten-fortsetzung.jpg}
				
			\paragraph{7. Erfahrung:} Kenntnisse und Erfahrungen beeinflussen, wie Elemente gruppiert werden
				\figwithcap{0.5}{res/gesetz-der-erfahrung.jpg}{\textit{Wir wissen aus Erfahrung, wie eine Birne aussieht, deshalb werden die Punkte als Birne wahrgenommen}}
	
		\pagebreak
	
	\subsection{Gestaltung mit Farben}
		\subsubsection{Salienz von Farbe}
			Farbe ist salient (sticht hervor). Dabei ist sparsamer Einsatz oft effektiver
			\fig{0.5}{res/salienz-von-farbe.jpg}
			
			Blau ist Konvention (s. z.B. Links bei Wikipedia u.a.).\\
			\textbf{Nachteil von Blau: } geringe Auflösung\\
			\textbf{Vorteil: } Nur wenige Menschen haben eine Blauschwäche\\
			
		\subsubsection{Theorien der Farbwahrnehmung}
			\begin{itemize}
				\item Dreifarbentheorie von Young/Helmholtz: Jeder Farbton kann aus 3
				monochromatischen Farben gemischt werden
					\fig{0.4}{res/dreifarbentheorie.jpg}
					
				\item Gegenfarbentheorie: Jeweils zwei Farben sind antagonistisch (in der Abb. untereinander); macht sich z.B. bei Nachbildern bemerkbar (s. USA-Flagge i.d. Abb.)
					\fig{0.5}{res/gegenfarbentheorie.jpg}
			\end{itemize}
		
		\pagebreak
	
		\subsubsection{Effektive Farbschemata}
			\paragraph{1. Triadic} basierend auf trichromatischer Wahrnehmung
				\fig{0.5}{res/triadic-color-scheme.jpg}
			
			\paragraph{2. Compound} basierend auf Gegenfarben	
				\fig{0.5}{res/compound-color-scheme.jpg}
				
			\paragraph{3. Analogous} basierend auf Farbverwandschaft
				\fig{0.5}{res/analogous-color-scheme.jpg}
				
			Tipp: color.adobe.com
				
		\pagebreak
			
		\subsubsection{Gestaltung für Farbfehlsichtigkeit}
			\begin{enumerate}
				\item Wichtige Informationen über mehrere Dimensionen kodieren (Symbole, räuml. Verteilung, Texturen)
				
				\item Farbpalette beschränken (duh!)
				
				\item Helligkeit z.B. bei hovers variieren
				
				\item Man kann auch Simulationstools benutzen (bspw. ColorOracle, http://colororacle.org/) \biglb
			\end{enumerate}
		
	
	\subsection{Gestaltung mit Bewegung}
		\subsubsection{Funktion der Bewegungswahrnehmung}
			\begin{itemize}
				\item Erregt Aufmerksamkeit
				\item Erzeugt realistisch und belebt wirkende Objekte
				\item Erzeugt Illusion von Raum \& 3D-Objekten (Erzeugung von Tiefe z.B. durch Verdeckung,\\ Parallaxe, \ldots)
				\item Gruppiert Objekte perzeptuell in Raum und Zeit
			\end{itemize}
			
			
			
			
			
			
			
			
			
			
			
			
			
			
			
	
	
	
	
	
	
	